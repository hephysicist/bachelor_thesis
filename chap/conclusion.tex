\chapter*{Заключение}
В ходе работы исследовался прототип координатного детектора фотонов, который планируется использовать в составе системы измерения энергии пучков ускорителя ВЭПП-4М. Сначала был изучены теоретические основы метода измерения энергии по резонансной деполяризации частиц в ускорителе. На основе полученных знаний сформированы требования к координатному детектору, а также  пояснена мотивация применения именно микроструктурных детекторов в конкретном случае. 
\par Далее была описана конструкция прототипа детектора, его технические характеристики. После этого рассмотрены особенности сбора и обработки данных для извлечения из них информации о координатах ионизирующих частиц. На языке Python написана библиотека в которой реализовано чтение файлов сырых данных с детектора, вычитание пьедесталов, нахождение кластеров, определение координат кластера методом центра тяжести и другие функции. 
\par В финальной части работы исследовались характеристики детектора. Определен уровень шумов, значение которого составило около $6000 e^-$. Данная информация необходима для правильного выбора порогового уровня при разделении сигнала и шума. С помощью радиоактивного изотопа проведено исследование зависимости коэффициента усиления от напряжения на ускоряющей структуре. Экспериментальные данные подчиняются экспоненциальному тренду, что говорит о правильной работе ускоряющей структуры и считывающей электроники. 
\par На выведенном пучке ускорителя ВЭПП-4М были измерены эффективность регистрации и пространственное разрешение. Максимально достигнутое значение эффективности при напряжении на ускоряющей структуре 3450 В составило $96 \pm 1\%$. Пространственное разрешение измерено отдельно для каждой из координат: для вертикальной координаты $\sigma_y = 0.257\pm0.007$\,мм, для горизонтальной $\sigma_x = 0.44\pm0.01$\,мм. Эти значения меньше, чем теоретическое предсказание для считывающей схемы, работающей в режиме идентификации треков. Объясняется это успешным применением метода центра тяжести при нахождении координат частиц, которые вызвали срабатывание группы каналов. 
\par Таким образом, первые исследования характеристик детектора для установки <<Лазерный поляриметр>> показали, что его можно успешно применять в системе измерения энергии. 