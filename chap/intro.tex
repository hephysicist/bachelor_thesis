\section{Введение}
\label{sec:intro}
Развитие экспериментальных методов ядерной физики привело к появлению большого количества детектирующих систем для ядерных измерений. Отдельно стоит выделить координатные детекторы, по которым до сих пор ведутся активные исследования. Главными направлениями являются повышение эффективности регистрации и пространственного разрешения  \cite{shechtman}.\par
Широкое распространение новых материалов и методов их обработки многократно улучшило параметры имеющихся детектирующих устройств, а так же позволило создавать детекторы новых конструкций. Так в 1997 г. группа ученых из Европейского центра ядерных исследований (ЦЕРН) под руководством Ф.Саули успешно применила концепцию газового электронного умножения в микроструктурах для создания координатных детекторов, которые получили название <<GEM-детекторы>> или газовые электронные умножители \cite{sauli}. Их отличительными особенностями является сравнительная простота конструкции, коэффициент усиления вплоть до $10^6$, а так же высокая радиационная стойкость. Данный тип детекторов широко используется в таких экспериментах, как PHOENIX (Франция), COMPASS (Швейцария), а так же в составе детекторов LHSb, TOTEM (ЦЕРН) и КЕДР (ИЯФ СО РАН).\par
В рамках работ по усовершенствованию установки <<выведенный пучок>> коллайдера ВЭПП-4М  планируется установка дополнительного массива координатных детекторов, которые должны обладать низкой плотностью, высокой эффективностью регистрации, достаточными пространственным и временным разрешением. Для выполнения данной задачи было решено использовать GEM-детекторы \cite{Bobr}. Каждый из них необходимо протестировать непосредственно на выведенном пучке с целью определения эффективности регистрации треков. Дополнительным необходимо исследовать распределение заряда между массивами считывающих элементов детектора. Последний параметр является критическим при подборе питающих напряжений для ускоряющих структур.\par