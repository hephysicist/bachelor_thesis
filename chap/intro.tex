\chapter{Введение}
\label{sec:intro}
Развитие экспериментальных методов ядерной физики привело к появлению большого количества детектирующих систем. Отдельно стоит выделить координатные детекторы, по которым до сих пор ведутся активные исследования. Главными направлениями являются повышение эффективности регистрации и пространственного разрешения \cite{shechtman}.
\par Широкое распространение новых материалов и методов их обработки многократно улучшило параметры имеющихся детектирующих устройств, а так же позволило создавать детекторы новых конструкций. Так в 1997 г. группа ученых из Европейского центра ядерных исследований (CERN) под руководством Ф.~Саули успешно применила концепцию газового электронного умножения в микроструктурах для создания координатных детекторов, которые получили название <<GEM-детекторы>> или газовые электронные умножители \cite{sauli}. Их отличительными особенностями являются сравнительная простота конструкции, коэффициент усиления вплоть до $10^6$, а так же высокая радиационная стойкость. Данный тип детекторов широко используется в таких экспериментах, как PHOENIX (Франция), COMPASS (Швейцария), а так же в составе детекторов LHSb, TOTEM (ЦЕРН) и КЕДР (ИЯФ СО РАН).
\par В ИЯФ микроструктурные детекторы применяются не только в составе детекторов для экспериментов в ФЭЧ (КЕДР, СНД и КМД-3), но и в различных системах, связанных с ними. Одной из таких систем является установка <<лазерный поляриметр>>. В  основе её работы лежит предложенный в 1975 г. в ИЯФ метод резонансной поляризации \cite{bukin}. Применяется данная система для прецизионного измерения энергии на коллайдере ВЭПП-4М.
\par В рамках работ по усовершенствованию <<лазерного поляриметра>> планируется установить новый координатный детектор. Для выполнения данной задачи было решено использовать GEM-детекторы \cite{Bobr}. В ИЯФ существует возможность изготовления таких детекторов с использованием GEM-электродов, производимых в CERN. Таким образом, возникает необходимость в исследовании новых моделей GEM-детекторов.
\par \textbf{Целью данной работы} являлось создание и исследование характеристик GEM-детектора для установки <<лазерный поляриметр>>. Понимание физических процессов работы детектирующей системы, организацию модуля сбора данных, а также особенностей их анализа дает наиболее полную информацию о точности измерений.  
Для достижения поставленной цели были сформулированы основные задачи, которые определили ключевые направления деятельности:
\begin{itemize}
    
    \item Изучение физических основ работы газовых электронных умножителей и основных схем GEM-детекторов
    \item Определение основных параметров, влияющих на коэффициент усиления детектора 
    \item  Установка, настройка и управление механизацией детектора
    \item Создание и отладка системы сбора и обработки данных.
    \item Проведение экспериментов на выведенном пучке, в ходе которых исследованы физические характеристики детектора
    \item Обработка и анализ полученных данных 
	
\end{itemize}