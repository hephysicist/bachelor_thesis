\chapter*{Введение}
\label{sec:intro}
Развитие экспериментальных методов ядерной физики привело к появлению большого количества детектирующих систем. Отдельно стоит выделить координатные детекторы, по которым до сих пор ведутся активные исследования. Главными направлениями являются повышение эффективности регистрации и пространственного разрешения \cite{shechtman}.
\par Широкое распространение новых материалов и методов их обработки многократно улучшило параметры имеющихся детектирующих устройств, а так же позволило создавать детекторы новых конструкций. Так в 1997 г. группа ученых из Европейского центра ядерных исследований (CERN) под руководством Ф.~Саули успешно применила концепцию газового электронного умножения в микроструктурах для создания координатных детекторов, которые получили название <<GEM-детекторы>> или газовые электронные умножители \cite{sauli}. Их отличительными особенностями являются сравнительная простота конструкции, коэффициент усиления вплоть до $10^6$, а так же высокая радиационная стойкость. Данный тип детекторов широко используется в таких экспериментах, как PHOENIX (Франция), COMPASS (Швейцария), а так же в составе детекторов LHSb, TOTEM (ЦЕРН) и КЕДР (ИЯФ СО РАН).
\par В ИЯФ микроструктурные детекторы применяются не только в составе детекторов для экспериментов в ФЭЧ (КЕДР, СНД и КМД-3), но и в различных системах, связанных с ними. Одной из таких систем является установка <<Лазерный поляриметр>>. В основе её работы эффект асимметрии обратного комптоновского рассеяния фотонов на поляризованных пучках. Применяется данная система для прецизионного измерения энергии коллайдера ВЭПП-4М \cite{nikitin-nikolaev}.
\par Экспериментальная программа комплекса ВЭПП-4М -- КЕДР включает в себя измерения масс и лептонных ширин $\Upsilon$--мезонов. Для этого необходимы калибровки абсолютного значения энергии пучков. Существующие на данный момент системы обладают малой эффективностью при энергиях, необходимых для рождения $\Upsilon$--мезонов. Поэтому планируется модернизировать <<Лазерный поляриметр>> путем установки нового координатного детектора. Следовательно, возникает потребность в исследовании его параметров и определении рабочих режимов.
\par \textbf{Целью данной работы} являлось создание и исследование характеристик GEM-детектора для установки <<лазерный поляриметр>>. Понимание физических процессов работы детектирующей системы, особенностей сбора данных, а также их анализа дает наиболее полную информацию о точности измерений.  
Для достижения поставленной цели были сформулированы основные задачи, которые определили ключевые направления деятельности:
\begin{itemize}
    
    \item Изучение физических основ работы газовых координатных детекторов и основных схем GEM-детекторов
    \item Создание и отладка системы сбора и обработки данных.
    \item Исследование уровня шумов, коэффициента усиления, эффективности регистрации и пространственного разрешения детектора 
	
\end{itemize}