\thispagestyle{empty}
\noindent 
\setstretch{1.2}
\begin{center}
{\large{Аннотация\\ к квалификационной работе Захарова Степана Алексеевича:\\Исследование физических характеристик прототипа GEM детектора для <<Лазерного поляриметра>> коллайдера ВЭПП-4М}}
\end{center}
\normalsize
\setstretch{1.2}
\vspace{10pt}
\\

В работе представлено исследование GEM--детектора в рамках проводимой модернизации установки «Лазерный поляриметр», использующей эффект асимметрии обратного комптоновского рассеяния для измерения энергии методом резонансной деполяризации. Проведен обзор теории поляризационных эффектов в пучках частиц, в ходе которого получена формула для связи энергии и асимметрии рассеяния фотонов в верхнее и нижнее полупространства. Определена относительная точность измерения энергии методом резонансной деполяризации, которая (для указанных в работе условий эксперимента) составила $2.57\cdot10^{-5}$. Рассмотрены принципы работы GEM--детекторов и определены параметры, которые необходимо исследовать: уровень шумов, коэффициент усиления, эффективность регистрации и пространственное разрешение. Измерение уровня шумов показало, что их среднее значение составляет примерно $6000~e^-$ что является вполне допустимым. Результаты измерений коэффициента усиления показали, что наблюдается его экспоненциальный рост с увеличением ускоряющего напряжения, что характерно для GEM--детекторов. Этим подтверждается корректная работа ускоряющей структуры и системы сбора данных. Измерения эффективности регистрации и пространственного разрешения проведены на выведенном пучке коллайдера ВЭПП-4М. Значение пространственного разрешения по вертикали составило: $\sigma_y = 0.258\pm0.007(\text{стат.})\pm0.017(\text{сист.})$\,мм, по горизонтали: $\sigma_x = 0.44\pm0.01(\text{стат.})$\,мм. Результаты лежат ниже теоретического предсказания, которое предполагало регистрацию координаты по каналу с максимальным зарядом. Это объясняется тем, что при анализе данных с детектора применен метод центра тяжести для определения координат частиц. По результатам исследования, проведенного в рамках бакалаврской работы, можно сделать вывод о корректной работе прототипа детектора и соответствию его параметров требуемым для <<Лазерного поляриметра>>.
\vspace*{20pt}

\flushright{\underline{\hspace*{4cm}} Захаров С.А.
\flushright \newline 11.06.2019 г.}