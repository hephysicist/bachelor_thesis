\section{Предел Рейтера и его зависимость от электрического поля в индукционном промежутке}
\label{sec:raether_exp}
Существенным параметром для детекторов, построенных с использованием ГЭУ, является коэффициент усиления. Существует два основных пути достижения его требуемых значений: 
\begin{itemize}
	\item обеспечение высокой электрической прочности электрода ГЭУ и приложение к нему более высоких напряжений
	\item использование несколько последовательно расположенных ГЭУ
\end{itemize}
Каждый из них имеет свои ограничения на применимость. ТУТ ОПИСАТЬ ПРО ПЕРВЫЙ МЕТОД.\par
Последовательное расположение нескольких ГЭУ с одной стороны вызывает усложнение конструкции детектора и увеличение количества материала на пути частицы. С другой стороны, напряжение на каждом электроде будет существенно ниже, чем в случае с одним ГЭУ. Нами было выдвинуто предположение