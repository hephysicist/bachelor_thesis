\section{Резонансная деполяризация и её применение для определения энергии пучка}
\label{sec:respnant_dep}
Эффект самопроизвольной поляризации заряженных частиц в ускорителях был описан Соколовым и Терновым еще в 1963г \cite{SokolovTernov63}. Качественно данный эффект можно описать следующим обоазом: в магнитном поле $\boldsymbol{H}$ потенциальная энергия частицы с магнитным моментом  $\boldsymbol{H}$ выражается как: 
\begin{equation}
U = - (\boldmath{\mu}, \boldmath{H}).
\end{equation} 
В случае поляризации пучка в ускорителе, $\boldsymbol{H}$ есть ведущее поле. Минимум потенциальной энергии дает значение угла между магнитным моментом и ведущим полем, равное нулю. Магнитный момент и спин электрона противоположно направлены, следовательно состояние электрона в пучке, в котором спин и магнитное поле антипараллельны, более устойчиво.
\par В работе  \cite{SokolovTernov63} определены доли от общего числа электронов, имеющие поляризацию по и против направления поля: 
\begin{multicols}{2}
	\begin{equation}
	n_{\uparrow\uparrow} = \frac{15+8\sqrt{3}}{30} \approx 0.962 
	\end{equation}
	
	\begin{equation}
	n_{\uparrow\downarrow} = \frac{15-8\sqrt{3}}{30} \approx 0.038
	\end{equation}
\end{multicols}
Можно заметить, что практические все электроны в пучке имеют спин направленный против ведущего поля. 
\par Еще одним эффектом, возникающим при движении частиц со спином в магнитных полях, является прецессия вектора спина вокруг направления магнитного поля.
\begin{equation}
\Omega =  \omega_{s}\bigg(\frac{q'}{q_0}\frac{E}{mc^2}+1\bigg)
\end{equation}