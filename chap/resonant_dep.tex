\chapter{Поляризационные эффекты и их применение для определения энергии пучка}
\label{sec:respnant_dep}
Эффект самопроизвольной поляризации заряженных частиц в ускорителях был описан Соколовым и Терновым еще в 1963г \cite{SokolovTernov63}. Качественно данный эффект можно описать следующим обоазом: в магнитном поле $\boldsymbol{H}$ потенциальная энергия частицы с магнитным моментом  $\boldsymbol{H}$ выражается как: 
\begin{equation}
U = - (\boldmath{\mu}, \boldmath{H}).
\end{equation} 
В случае поляризации пучка в ускорителе, $\boldsymbol{H}$ есть ведущее поле. Минимум потенциальной энергии дает значение угла между магнитным моментом и ведущим полем, равное нулю. Магнитный момент и спин электрона противоположно направлены, следовательно состояние электрона в пучке, в котором спин и магнитное поле антипараллельны, более устойчиво.
\par В работе  \cite{sokolov} определены доли от общего числа электронов, имеющие поляризацию по и против направления поля: 
\begin{multicols}{2}
	\begin{equation}
	n_{\uparrow\uparrow} = \frac{15+8\sqrt{3}}{30} \approx 0.962 
	\end{equation}
	
	\begin{equation}
	n_{\uparrow\downarrow} = \frac{15-8\sqrt{3}}{30} \approx 0.038
	\end{equation}
\end{multicols}
Можно заметить, что практические все электроны в пучке имеют спин направленный против ведущего поля. 
\par Еще одним эффектом, возникающим при движении частиц со спином в магнитных полях, является прецессия спина вокруг направления ведущего поля. Частота прецессии определяется формулой \ref{Omega} и зависит от энергии, соотношения между аномальной и нормальной частью гиромагнитного отношения и частотой вращения пучков в ускорителе. 
\begin{equation}
\omega_s=  \omega_{r}\bigg(\frac{q'}{q_0}\frac{E}{mc^2}+1\bigg) \label{Omega}
\end{equation}
Идея метода резонансной деполяризации, предложенного в ИЯФ СО РАН в 1974 г. подробно описывается в \cite{skrinskii}. Его суть заключается в том, что на пучок частиц воздействует переменное высокочастотное электромагнитное поле. При достижении резонансного условия: 
\begin{equation}
\omega_s=  k\omega_{r} \pm \omega_d,
\end{equation}
где $\omega_d$ -- частота электромагнитного поля, исходная поляризация пучка нарушается, что можно определить например оптическим методом. Проводя сканирование по $\omega_d$ и фиксируя момент деполяризации, можно достаточно точно определить $\omega_s$.
\par Приведем оценку точности данного метода. 
\begin{itemize}
	\item $\displaystyle \frac{\delta (q'/q_0)}{q'/q_0} = $
	\item $\displaystyle \frac{\delta (mc^2)}{mc^2}= 6.06\cdot10^{-9} $
	\item $\displaystyle \frac{\delta (\omega_r)}{\omega_r} = 0.???? $
\end{itemize}
