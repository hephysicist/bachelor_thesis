\chapter{Поляризационные эффекты и их применение для определения энергии пучка}
\label{sec:respnant_dep}
\section{Радиационная поляризация}
Эффект самопроизвольной поляризации заряженных частиц в ускорителях был описан Соколовым и Терновым еще в 1963г \cite{sokolov}. Качественно данный эффект можно описать следующим обоазом: в магнитном поле $\vec{H}$ потенциальная энергия частицы с магнитным моментом  $\vec{\mu}$ выражается как: 
\begin{equation}
U = - (\vec{\mu}, \vec{H}).
\end{equation} 
В случае поляризации пучка в ускорителе, $\vec{H}$ есть ведущее поле. Минимум потенциальной энергии дает значение угла между магнитным моментом и ведущим полем, равное нулю. Магнитный момент и спин электрона противоположно направлены, следовательно состояние электрона в пучке, в котором спин и магнитное поле антипараллельны, более устойчиво.
\par В работе  \cite{sokolov} определены доли от общего числа электронов, имеющие поляризацию против и по направлению поля: 
\begin{multicols}{2}
	\noindent
	\begin{equation}
	n_{\uparrow\downarrow} = \frac{15+8\sqrt{3}}{30} \approx 0.962 
	\end{equation}
	\begin{equation}
	n_{\uparrow\uparrow} = \frac{15-8\sqrt{3}}{30} \approx 0.038
	\end{equation}
\end{multicols}%
\noindent Можно заметить, что практические все электроны в пучке имеют спин, направленный против ведущего поля.
\par Процесс поляризации пучков в ускорителях может занимать от десятков минут до сотен часов. Временная зависимость поляризации пучка задается формулой: 
\begin{equation}
	\mathcal{P} = G\zeta_0(1-e^{-t/G\tau_p}),
	\label{eq:polDepOnTime}
\end{equation}
где $G$ -- деполяризующий фактор, $\zeta_0 = 8/(5\sqrt{3})$ -- максимально достижимое значение поляризации пучка. $\tau_p$ есть время поляризации, которое зависит от параметров ускорителя (радиуса орбиты $R$ и энергии $E$). Полное выражение для него получили Соколов и Тернов:
\begin{equation}
\tau_p = \biggl[\cfrac{5 \sqrt{3}}{8} \cfrac{e^2\hbar}{m_e ^2c^2 R^3} \biggl(\cfrac{E}{m_ec^2}\biggr)^5 \biggr]^-1 \propto \cfrac{1}{E^5}
\label{eq:pol_time}
\end{equation} 
Время поляризации обратно пропорционально пятой степени энергии. Например, в ВЭПП-4М на энергии $4~\GeV$ время поляризации --- величина порядка часа, что позволяет определять энергию пучков методом резонансной деполяризации.
\section{Метод резонансной деполяризации}
Еще одним эффектом, возникающим при движении частиц со спином в электромагнитных полях, является прецессия спина $\vec{S}$ вокруг направления ведущего поля $\vec{H}$. Уравнение движения спина:
\begin{equation}
\cfrac{d\vec{S}}{dt} = [\vec{\Omega},\vec{S}],
\label{eq:precession_full}
\end{equation}
где $\vec{\Omega}$ имеет следующий вид:
\begin{equation}
\vec{\Omega} = -\biggl(\frac{q_0}{\gamma}+q'\biggr) \vec{H} + \frac{\gamma}{\gamma + 1}q' (\vec{\upsilon},\vec{H})\vec{\upsilon}- \biggl(\frac{q_0}{\gamma+1}  + q'\biggr)[\vec{\mathcal{E}},\vec{\upsilon}].
\end{equation}%
Здесь $q_0$ и $q'$ соответственно нормальная и аномальная части гиромагнитного отношения, $\gamma$ --- релятивистский гамма--фактор, $\vec{\upsilon}$ --- скорость частицы, $\vec{\mathcal{E}}$ --- вектор электрического поля.
В наиболее простом случае, когда $(\vec{\upsilon}, \vec{H})$ и $[\vec{\mathcal{E}},\vec{\upsilon}]$ равны нулю, имеем только один член, определяющий прецессию спина в ведущем магнитном поле. Таким образом, уравнение \ref{eq:precession_full} принимает вид:
\begin{equation}
\cfrac{d\vec{S}}{dt} = -\biggl(\frac{q_0}{\gamma}+q'\biggr) [\vec{H}, \vec{S}].
\label{eq:precession}
\end{equation}%
Если выразить ведущее поле через частоту обращения пучков как: 
\begin{equation}
\vec{H} = \frac {\gamma mc}{e}\omega_r\vec{n}_H.
\label{eq:larmor_freq}
\end{equation}
Подставим \ref{eq:larmor_freq} в \ref{eq:precession} и проведем серию математических преобразований чтобы получить выражение для частоты прецессии спина:
\begin{equation}
\omega_s=  \omega_{r}\bigg(\frac{q'}{q_0}\frac{E}{mc^2}+1\bigg).
\label{eq:spin_freq}
\end{equation}
Если измерить $\omega_s$ и $\omega_{r}$, то можно определить энергию электрона $E$ т.к. остальные константы в выражении \ref{eq:spin_freq} известны. $\omega_{r}$ можно найти разными способами: прямым измерением с помощью мониторов пучка, по частоте ускоряющего поля в резонаторе и т.д. Однако, определение $\omega_s$ является весьма нетривиальной задачей. 
\par Один из методов, с помощью которого можно косвенно измерить  $\omega_s$ по регистрации резонансной деполяризации предварительно поляризованного пучка частиц, был разработан в ИЯФ СО РАН в 1974 г. и детально описывается в \cite{MRD}. Идея метода заключается в воздействии на пучок переменного электромагнитного поля определенной частоты. Если выполняется резонансное условие:
\begin{equation}
\omega_s=  k\omega_{r} \pm \omega_d,
\end{equation}
где $\omega_d$ -- частота электромагнитного поля, а $k$ -- произвольное целое число,  то исходная поляризация пучка нарушается. Это можно определить любым поляризационно чувствительным методом. Проводя сканирование по $\omega_d$ и фиксируя момент деполяризации, можно определить $\omega_s$.
\par Приведем оценку точности данного метода. Для этого определим какова точность определения параметров, входящих в выражение \ref{eq:spin_freq}:
\[
	\begin{matrix}
	  \displaystyle \frac{\delta (q'/q_0)}{q'/q_0} = 2.24 \cdot 10^{-10} \quad
	& \displaystyle \frac{\delta (mc^2)}{mc^2}= 6.06\cdot10^{-9} \quad
	& \displaystyle \frac{\delta (\omega_r)}{\omega_r} = 10^{-10}.
	\end{matrix}
\]
\noindent Можно заметить, что точность определения массы электрона вносит наибольший вклад в точность измерения энергии. Физическое ограничение для $\delta E/E$ устанавливается на уровне $10^{-8}$. Измерения, использующие метод резонансной деполяризации, являются на данный момент самыми точными в мире.
\section{Регистрация эффекта деполяризации}
Существует несколько методов, с помощью которых можно зарегистрировать момент деполяризации. Все они предполагают рассеяние электронов пучка на ядрах (<<моттовское>> рассеяние), на электронах этого же пучка (<<тушековское>> рассеяние), а также <<комптоновское>> рассеяние на поляризованных фотонах. Первый метод малоэффективен ввиду отсутствия в вакуумной камере достаточного количества атомов, на которых рассеивался бы пучок. Установки, использующие эффект <<тушековского>> рассеяния широко используются на малых энергиях ($E < 2~\GeV$). Из-за того, что измеряемый эффект обратно пропорционален четвертой степени энергии пучка, то в области $\Upsilon$---резонанса его применение является малоэффективным.
\par В данном случае Комптоновское рассеяние поляризованных фотонов на пучке электронов является единственным методом, применимым в данном диапазоне энергий. Сечение рассеяния зависит как от поляризации электрона, так и от поляризации фотона. Идея метода заключается в следующем: если пучок электронов поляризован, то существует связь между направлением рассеяния фотонов и их поляризацией. При рассеянии циркулярно поляризованного света на вертикально поляризованном пучке наблюдается асимметрия рассеяния в верхнюю и нижнюю полусферу. В \cite{landau_4} получена формула для дифференциального сечения рассеяния поляризованных фотонов на поляризованных электронах:
\begin{equation}
d \sigma(\vec{\xi}, \vec{\zeta}) = d\sigma(\vec{\xi}) + \frac{1}{2}r_e^2 \biggl(\frac{\omega'}{\omega}\biggr)^2 V (\vec{f},\vec{\zeta})do',
\label{eq:diff_cross_sec}
\end{equation}
где $d\sigma(\vec{\xi})$ есть сечение рассеяния фотонов на неполяризованных электронах, $\vec{\xi}$ и $\vec{\zeta}$ -- векторы поляризации фотонов и электронов соответственно, $\omega$ и $\omega'$ -- частоты падающего и рассеявшегося фотонов, $r_e$ -- классический радиус электрона, V -- параметр Стокса (циркулярность поляризации фотона), $\vec{f}$ -- вектор, зависящий от направлений волновых векторов падающего ($\vec{k}$) и рассеявшегося (${k'}$) фотонов: 
\begin{equation}
	\vec{f} = \frac{1 - \cos(\theta)}{m_e}\bigg[\vec{k} \cos(\theta) + \vec{k'}\bigg]
\end{equation}
 По зависимости сечения от параметров падающего фотона можно заметить, что асимметрия рассеяния возникает только при ненулевом параметре $V$. Значит возникает необходимость использовать циркулярно поляризованные лазерные пучки. 
 \par В таком случае можно из уравнения \ref{eq:diff_cross_sec} получить формулу для определения измеряемого эффекта асимметрии рассеяния, то есть, расстояния между центрами распределений по вертикальной координате отраженных фотонов с разными циркулярностями поляризации:
\begin{equation}
	\Delta y = \frac{\hbar \omega_0}{2 m_e c^2} \mathcal{P} \Delta V L,
	\label{eq:pol_effect}
\end{equation}
где $\omega_0$ -- частота падающего фотона, $\mathcal{P}$ -- поляризация электронного пучка, $\Delta V$ -- разница параметров Стокса для циркулярно поляризованного пучка фотонов, $L$ -- расстояние от точки взаимодействия до точки регистрации рассеянного фотона. 
\begin{figure}[H]
	\begin{center}
		\includegraphics[width = 8cm]{img/mrd-lsrp.png}
		\caption{Принципиальная схема установки по регистрации поляризации электронного пучка. Пучок заряженных частиц (электронов) облучается циркулярно поляризованными фотонами. Происходит обратное комптоновское рассеяние фотонов на электронах, в результате чего образуются высокоэнергитичные (до $1~\GeV$) гамма--кванты, которые регистрируются координатным детектором. Измеряемый эффект есть вертикальное расстояние ($\Delta y$) между центрами распределений, полученных от фотонов левой и правой циркулярной поляризации.}
		\label{fig:laser_polarimeter_scheme}
	\end{center}
\end{figure}
\vspace{-20pt}
При деполяризации электронного пучка $\mathcal{P} = 0$, следовательно вертикальная асимметрия рассеяния фотонов пропадает. Чтобы это зарегистрировать, необходим детектор с возможностью регистрации вертикальной координаты. В случае поляризованного электронного пучка распределения по координате фотонов для левой и правой циркулярной поляризаций будут смещены друг относительно друга. Деполяризация пучка будет выглядеть как слияние двух распределений в одно.\vspace{-10pt}

\section{Оценки эффекта и точности определения энергии}
<<Лазерный поляриметр>> коллайдера ВЭПП-4М имеет следующие параметры: $\lambda = 527$\,нм , $L = 40$\,м , $\Delta V = 2$. Подставив их в формулу \ref{eq:pol_effect}, можем получить оценку измеряемого эффекта $\Delta y$, которая составляет около  0.1 мм. Измерение такого малого смещения, на первый взгляд, представляется затруднительным, однако регистрация большого количества фотонов и определение средней вертикальной координаты по выборке имеет $1/\sqrt{N}$ меньшую статистическую погрешность, но в таком случае придется учитывать еще разброс энергий  и вертикальных компонент импульсов электронов в пучке. Стоит заметить, что координатное разрешение детектора в данном случае может быть и больше величины эффекта, но определение среднего по выборке даст требуемую точность. 
\par Чтобы оценить погрешность определения энергии рассмотрим, как проводится процесс измерения абсолютного значения энергии, концептуально описанный выше. Сканирование по частоте деполяризации, а значит по энергии проводится со скоростью -- $\dot{E}$. В каждой точке за время $T$ набирается суммарная статистика -- $N$, которая представляет собой два набора по $N/2$ координат фотонов с левой и правой циркулярной поляризацией. Погрешности определения координат $\sigma_x$ и $\sigma_y$  в основном зависят от типа детектирующей системы. Для каждого набора определяется среднее значение. Разница между средними значениями вертикальных координат двух наборов (то есть, между центрами пятен в детекторе) и является измеряемым эффектом $\Delta y$. 

 Погрешность измерения $\Delta y$ можно выразить следующим образом: 
\begin{equation}
	\delta(\Delta y) = \sqrt{(\delta y_{up})^2 + (\delta y_{down})^2} = \frac{\sqrt{2} \sigma_y}{\sqrt{N/2}},
\end{equation}
где $\delta y_{up}$ и $\delta y_{down}$ погрешность определения центров пятен.
Теперь удобно сделать замену: $N = \dot{N}T$ потому что $\dot{N}$ является параметром системы сбора данных. Измерение эффекта деполяризации фиксируется по слиянию двух пятен в детекторе. 

На Рис. \ref{fig:RD_effect_graph} показан вид экспериментальных данных с предыдущей системы калибровки абсолютного значения энергии, полученных методом резонансной деполяризации, в области массы $\tau$--лептона. В качестве координатного детектора использовался детектор для установки <<ДЕЙТОН>>.
\begin{figure}[H]
	\begin{center}
		\includegraphics[width = 16cm]{img/rd_exp.pdf}
		\caption{График измерения абсолютного значения энергии в области $\tau$--лептона. $\delta t$ есть ошибка определения времени деполяризации, а $\Delta y$ -- измеряемый эффект.}
		\label{fig:RD_effect_graph}
	\end{center}
\end{figure}
\vspace{-20pt}
\noindent Определим погрешность измерения энергии как: 
\begin{equation}
\delta(E)= \dot{E} \delta t = \cfrac{\delta(\Delta y)}{\Delta y}T\dot{E} = \frac{2 \sigma_y}{\sqrt{\dot{N}} \Delta y}\sqrt{T}\dot{E}  
\end{equation}
Видим, что $\delta E$ обратно пропорциональна корню из скорости набора статистики. За одну вспышку лазера по расчетам на пучке будет происходить рассеяние порядка 10 фотонов на 1 мА тока. На энергии пучка $4~\GeV$ возможно обеспечить ток в 10 мА. Это значит, что при успешной регистрации половины фотонов на частоте $4~\kHz$ и времени эксперимента $ T = 300$\,с значение погрешности измерения энергии $\delta(E)= 108~\keV$.
 
