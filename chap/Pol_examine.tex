\section{Исследование прототипа детектора для установки <<Лазерный поляриметр>>}
\label{sec:pol_examine}
\subsection{Конструкция детектора}
Для регистрации одиночных гамма--квантов, полученных обратным комптоновским рассеянием на пучках электронов, был спроектирован и изготовлен прототип детектора, использующего ГЭУ для усиления сигнала первичной ионизации. В конструкции применен тройной электрод с питанием от резистивного делителя. Основа детектора представляет собой многослойную плату из СТЭФ с массивом плоских металлических электродов, расположенных в её центральной части, которую можно видеть на Рис. \ref{fig:Pol_det_photo} 
\par Электроды поделены на группы, каждая из которых регистрируется отдельной считывающей платой. Всего считывающих плат десять, и они установлены в специальные многоканальные разъемы на периферии основной платы. Front-end электроника включает в себя быстрые АЦП и ПЛИС для работы с ними. Далее сигнал по USB подается на компьютер. На данный момент разрабатывается программное обеспечивающее взаимодействие всех 10 плат и одновременное считывание события с детектора. Решено было использовать для последующих экспериментов только одну из плат, т.к. вычитывание данных с неё уже отлажено. 
\par Электроды ГЭУ укреплены на рамках из 1.5 мм СТЭФ над считывающей структурой. Сверху на плату закрепляется герметичный кожух из СТЭФ с трубками для ввода и вывода газовой смеси. Сборка детектора осуществляется в корпусе из листового алюминия.
\par Данный детектор отличается от аналогов большей гранулярностью, что даст большую точность определения координат событий. (может ещё что написать)
\subsection{Особенности сбора и обработки сырых данных}
Каналы детектора объединены в группы по 100 (центральная часть) или 120 (периферия) каналов. Каждая группа скоммутирована на отдельный разъем, к которому подключены два многоканальных АЦП. Одновременно можно вычитывать данные со всех каналов. При поступлении сигнала с триггера, схема начинает последовательно раз в 125 нс вычитывать заряд со всех каналов. Таким образом вычитывание происходит 100 раз. Каждый отсчет времени будем в дальнейшем называть <<кадром>>, а массив данных о заряде для каждого из каналов группы и каждого кадра из 100 назовем событием. 
\par События, каждое из которых состоит из 12800 целых чисел, записываются в файл формата txt. Формат вывода следующий: одна строка соответствует одному кадру и состоит из 128 чисел. Таких строк 100. 101--я строка содержит номер кадра, с которого началось вычитывание значений АЦП. На самом деле микросхемы АЦП непрерывно вычитывают заряд с каналов, но ПЛИС возвращает событие только при активации триггера. 
\par Из-за технических особенностей схемы front-end электроники нулевой уровень сигнала находится в районе 7400 каналов АЦП. Сигнальное событие представляет собой импульс отрицательной полярности, который имеет резкий передний фронт (1 кадр) и экспоненциально затухающий задний фронт (3-10 кадров в зависимости от суммарного заряда).